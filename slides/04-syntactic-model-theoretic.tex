\documentclass{beamer}
\usepackage{booktabs}
\usepackage{pdfpages}
\usepackage{mathtools}
\usepackage{enumerate}
\usepackage{multirow,tabularx}
\usepackage{booktabs}
\usepackage{pdfpages}
\usepackage{proof}
\usepackage{cancel}
\usepackage{chronology}
\usepackage{graphicx}
\usepackage{ulem}
\usepackage{amsmath}
\usepackage{amssymb}
\usepackage{color}
\usepackage{animate}
\usepackage{xr}

\PassOptionsToPackage{usenames,dvipsnames,svgnames}{xcolor}  
\usepackage{tikz}
\usepackage{tkz-graph}


\usepackage{wasysym}
\usepackage{proof}
\usepackage{cancel}
\usepackage{chronology}
\usepackage{graphicx}
\usepackage{ulem}
\usepackage{amsmath}
\usepackage{amssymb}
\usepackage{color}
\usepackage{xcolor}
\usepackage{soul}
%\usepackage{pstricks}
\setbeamertemplate{navigation symbols}{}

\newcommand{\norm}[1]{\left\lVert#1\right\rVert}
\newcommand{\el}{$\mathcal{EL}^{++}$}
\renewcommand{\Re}{\mathbb{R}}
\newcommand{\BigO}[1]{\ensuremath{\operatorname{O}\bigl(#1\bigr)}}
\newcommand{\myul}[2][blue]{\sethlcolor{#1}\hl{#2}\setulcolor{black}}

\newcommand<>{\cunderline}[3]{\only<#1>{#3}\only<#2>{\underline{#3}}}
\newcommand<>{\cem}[3]{\only<#1>{#3}\only<#2>{\ul{#3}}}
\newcommand<>{\cgray}[3]{\only<#1>{#3}\only<#2>{\textcolor{gray}{#3}}}
\newcommand<>{\colorize}[4]{\only<#1>{#4}\only<#2>{\textcolor{#3}{#4}}}

\setbeamertemplate{navigation symbols}{\insertslidenavigationsymbol}
%\setbeamertemplate{navigation symbols}{}
% \addtobeamertemplate{navigation symbols}{}{%
%     \usebeamerfont{footline}%
%     \usebeamercolor[fg]{footline}%
%     \hspace{1em}%
%     \insertframenumber/\inserttotalframenumber
% }

\mode<presentation>
{
\usecolortheme{crane}
%\useoutertheme{split}

\expandafter\def\expandafter\insertshorttitle\expandafter{%
  \insertshorttitle\hfill%
  \insertframenumber\,/\,\inserttotalframenumber}

\usefonttheme[onlysmall]{structurebold}
}
\renewcommand{\em}{\itshape}
\usepackage{pifont}
\definecolor{purple}{rgb}{1,0,1}
\definecolor{dred}{rgb}{0.7,0,0}
\definecolor{myred}{rgb}{1,0,0}
\definecolor{dblue}{rgb}{0,0,0.7}
\definecolor{dgreen}{rgb}{0,0.5,0}
\definecolor{myyellow}{rgb}{1,1,0}
\newcommand{\parents}[1]{parents(#1)}
\setbeamertemplate{itemize item}[ball]


% \mode<presentation>
% {
% \useinnertheme[shadow=true]{rounded}
% \useoutertheme{infolines}
% \usecolortheme{dove}
% \setbeamerfont{block title}{size={}}
% }

\title[Bio-Ontologies]{Relational data and knowledge}

%\author{Robert Hoehndorf and Maxat Kulmanov}


\date{}

\begin{document}

\begin{frame}
  \titlepage
\end{frame}

\section{Embedding ontologies}

% \begin{frame}
%   \frametitle{Embedding ontologies: approaches}
%   \begin{itemize}
%   \item syntactic: treat axioms as ``sentences'' using language models
%   \item graph-based: treat ontologies as graphs (like in semantic similarity)
%   \item model-theoretic: encode model-theoretic semantics in optimization
%   \end{itemize}
% \end{frame}

\subsection{Syntactic approaches}

\begin{frame}
  \frametitle{Ontologies: axioms, not graphs!}
    \includegraphics[width=1\textwidth]{bcellapoptosis.png}
\end{frame}

\begin{frame}
  \frametitle{Ontologies: axioms, not graphs!}
  Gene Ontology:
  \begin{itemize}
  \item {\tt behavior DisjointWith: 'developmental process'}
  \item {\tt behavior SubclassOf: only-in-taxon some metazoa}
  \item {\tt 'cell proliferation' DisjointWith: in-taxon some fungi}
  \item {\tt 'cell growth' EquivalentTo: growth and ('results in
      growth of' some cell)}
  \item ...
  \end{itemize}
\end{frame}

% \begin{frame}
%   \frametitle{Ontologies: axioms, not graphs!}
%   \begin{itemize}
%   \item converting ontologies to graphs
%     \begin{itemize}
%     \item loses information
%     \end{itemize}
%   \item relations between ontologies
%   \end{itemize}
% \end{frame}

% \begin{frame}
%   \frametitle{Ontology embeddings}
%   \begin{definition}
%     Let $O = (\Sigma = (C, R, I); ax; \vdash)$ be an ontology with a set of
%     classes $C$, a set of relations $R$, a set of instances $I$, a set
%     of axioms $ax$ and an inference relation $\vdash$. An ontology
%     embedding is a function $f_\eta : C \cup R \cup I \mapsto
%     \Re^n$ (or $\Sigma(O) \mapsto \Re^n$ (subject to certain constraints).
%   \end{definition}
%   \pause For example, we can use co-occurrence within $ax^\vdash$ to
%   constrain the embedding function, where the constraints on
%   co-occurrence are formulated using the Word2Vec skipgram model.
% \end{frame}

\begin{frame}
  \frametitle{Onto2Vec}
  \centerline{\includegraphics[width=\textwidth]{onto2vecflow.png}}
\end{frame}

% \begin{frame}
%   \frametitle{Word2Vec}
%   Maximize:
%   \begin{equation}
%     \frac{1}{N} \sum_{n=1}^{N} \sum_{-c\leq j \leq c, j\not=
%       0} \log p(w_{n+j}|w_n)
%   \end{equation}
%   with
%   \begin{equation}
%     p(w_o | w_i) = \frac{\exp({v'_{w_o}}^T v_{w_i})}{\sum_{w=1}^{W}
%       \exp({v'_{w}}^T v_{w_i})}
%   \end{equation}
%   (at least conceptually; different strategies are used to approximate Eqn. 2)
% \end{frame}

% \begin{frame}
%   \frametitle{Word2Vec}
%   \centerline{\includegraphics[width=\textwidth]{word2vec-example.png}}
% \end{frame}

% \begin{frame}
%   \frametitle{Predicting PPIs: trainable similarity measures}
%   \centerline{\includegraphics[width=.45\textwidth]{YSTUnsuper1.png}\includegraphics[width=.45\textwidth]{YstUnsup2.png}}

%   {\tiny Smaili et al. Onto2Vec: joint vector-based representation of
%      biological entities and their ontology-based annotations.}
% \end{frame}

\begin{frame}
  \frametitle{Visualizing embeddings}
  \centerline{\includegraphics[width=.9\textwidth]{updtsne.jpg}}
\end{frame}

\begin{frame}
  \frametitle{Combination with text}
  \begin{itemize}
  \item ontologies contain more than axioms:
    \begin{itemize}
    \item labels, synonyms, definitions, authors, etc.
    \end{itemize}
  \item Description Logic axioms != natural language
  \item transfer learning: learn on one domain/task, apply to another
    \begin{itemize}
    \item e.g.: learn on literature, apply to ontologies
    \item words have ``meaning'' in literature, Description Logic
      symbols have ``meaning'' in ontology axioms
    \end{itemize}
  \item Ontologies Plus Annotations 2 Vec (OPA2Vec) combines both
  \end{itemize}
\end{frame}

\begin{frame}
  \frametitle{Ontologies Plus Annotations 2 Vec}
  \centerline{\includegraphics[width=1\textwidth]{opaworkflow16.png}}
\end{frame}

% \begin{frame}
%   \frametitle{Axioms contribute to prediction tasks: GO and GO-PLUS}
%     % \processtable
%     % \caption{AUC values of ROC curves for PPI prediction for
%     %   GO-Plus and GO using Onto2Vec (cosine similiarity) and
%     %   Onto2Vec-NN (neural network).\label{Tab:GOplus}}
%     { \begin{tabularx}{\columnwidth}{XXXXXXX}\toprule & {}& {} &
%                                                                  Human & Yeast & Arabidopsis\\\midrule
%         $GO\_Onto2Vec$ & {} &{}& 0.7660 & 0.7701 & 0.7559 \\
%         $GO\_Onto2Vec\_NN$ & {} & {}& 0.8779& 0.8711 & 0.8364 \\
%         $GO\_plus\_Onto2Vec$ & {} & {}& 0.7880& 0.7943 & 0.7889 \\
%         $GO\_plus\_Onto2Vec\_NN$ &{} & {}& \textbf{0.9021}&\textbf{0.8937} & \textbf{0.8834}\\
%         \hline
%       \end{tabularx}}{}
% \end{frame}

% \begin{frame}
%   \frametitle{Evaluating individual axioms}
% %  Testing how much each ontologies' axioms contribute to predictions:
% %  \begin{resizebox}{\textwidth}{!}{
%   Testing how much each ontologies' axioms contribute to predictions:
%   \tiny
%       \begin{tabularx}{\linewidth}{X|XX|XX}
%         \toprule
%         {} & \multicolumn{2}{c}{\textbf{Human}} & \multicolumn{2}{c}{\textbf{Arabidopsis}}\\
%         \midrule
%         {} & \textbf{Onto2Vec}&\textbf{Onto2Vec\_NN} &\textbf{Onto2Vec}&\textbf{Onto2Vec\_NN} \\
%         \midrule
%         GO (Baseline) &0.7660 &0.8779  & 0.7559 & 0.8364 \\
%         ChEBI &0.7899\textcolor{blue}{(+0.0239)}& 0.8914\textcolor{blue}{(+0.0135)}  & 0.7703\textcolor{blue}{(+0.0144)}& 0.8518\textcolor{blue}{(+0.0154)} \\
%         PO &0.7752\textcolor{blue}{(+0.0092)} & 0.8776\textcolor{red}{(-0.0003)} & 0.7671\textcolor{blue}{(+0.0112)} & 0.8469\textcolor{blue}{(+0.0105)}\\
%         CL & 0.7743\textcolor{blue}{(+0.0083)} & 0.8810\textcolor{blue}{(+0.0031)} & 0.7612\textcolor{blue}{(+0.0053)}& 0.8371\textcolor{blue}{(+0.0007)}\\
%         PATO & 0.7657\textcolor{red}{(-0.0003)} & 0.8768\textcolor{red}{(-0.0011)} & 0.7563\textcolor{blue}{(+0.0004)} & 0.8380\textcolor{blue}{(+0.0016)}\\
%       \end{tabularx}
% %    }
% %  \centerline{\includegraphics[width=1\textwidth]{pato-eval1.png}}
% \end{frame}

% \begin{frame}
%   \frametitle{Evaluating definitions}
%   Testing how much each ontologies' annotation properties contribute to predictions:
%   \tiny
%       \begin{tabularx}{\linewidth}{X|XX|XX}
%         \toprule
%         {} & \multicolumn{2}{c}{\textbf{Human}} & \multicolumn{2}{c}{\textbf{Arabidopsis}}\\
%         \midrule
%         {} & \textbf{Onto2Vec}&\textbf{Onto2Vec\_NN} &\textbf{Onto2Vec}&\textbf{Onto2Vec\_NN} \\
%         \midrule
%         GO (Baseline)&0.8727 &0.9033  & 0.8613 & 0.8903 \\
%         ChEBI & 0.8571\textcolor{red}{(-0.0156)} &0.8801\textcolor{red}{(-0.0232)} &0.8601\textcolor{red}{(-0.0012)}& 0.8880\textcolor{red}{(-0.0023)}\\
%         PO & 0.8680\textcolor{red}{(-0.0047)}&0.8824\textcolor{red}{(-0.0209)} & 0.8632\textcolor{blue}{(+0.0019)} & 0.8908\textcolor{blue}{(+0.0005)}\\
%         CL & 0.8811\textcolor{blue}{(+0.0084)}&0.9037\textcolor{blue}{(+0.0004)}  & 0.8614\textcolor{blue}{(+0.0001)} & 0.8899\textcolor{red}{(-0.0004)}\\
%         PATO & 0.8562\textcolor{red}{(-0.0165)}& 0.8711\textcolor{red}{(-0.0322)} & 0.8544\textcolor{red}{(-0.0069)}& 0.8860\textcolor{red}{(-0.0043)} \\
%       \end{tabularx}
% %  \centerline{\includegraphics[width=1\textwidth]{pato-eval2.png}}
% \end{frame}

\begin{frame}
  \frametitle{Onto2Vec and OPA2Vec}
  \begin{itemize}
  \item \url{https://github.com/bio-ontology-research-group/mowl}
  \item python library
    \begin{itemize}
    \item input: OWL ontology, set of entities with annotations/associations
    \item output: vectors for each class and entity
    \end{itemize}
  \item Elk reasoner
  \item limitations: word-based
    \begin{itemize}
    \item completely ignores any semantics!
    \end{itemize}
  \end{itemize}
\end{frame}

\subsection{Model-theoretic approaches}

\begin{frame}
  \frametitle{How to overcome the semantic gap?}
  \begin{itemize}
  \item none of the models discussed above are truly ``semantic''
    \begin{itemize}
    \item all syntactic
    \item graph-based or based on axioms
    \end{itemize}
    \pause
  \item what do we actually mean by ``semantics''?
    \begin{itemize}
    \pause
    \item formal definition of ``truth'' relies on ``models''
    \pause
    \item universal algebra over formal languages (with signature
      $\Sigma$)
    \end{itemize}
  \end{itemize}
\end{frame}

\begin{frame}
  \frametitle{Description Logic EL++}
  \centering
  \resizebox{.8\textwidth}{!}{
    
    \begin{tabular}{|p{1.7cm}|c|p{3.7cm}|}
      \hline
      {\bf} Name & Syntax & Semantics \\
      \hline
      top & $\top$ & $\Delta^{\mathcal{I}}$ \\
    \hline
    bottom & $\bot$ & $\emptyset$ \\
    \hline
    nominal & $\{ a \} $ & $\{ a^{\mathcal{I}} \}$ \\
    \hline
    conjunction & $C \sqcap D$ & $ C^{\mathcal{I}} \cap
                                 D^{\mathcal{I}}$ \\
    \hline
    existential restriction & $\exists r.C$ & $ \{ x \in
                                              \Delta^{\mathcal{I}} |
                                              \exists y \in
                                              \Delta^{\mathcal{I}} :
                                              (x,y) \in
                                              r^{\mathcal{I}} \land y
                                              \in C^{\mathcal{I}} \} $
    \\
    \hline
    generalized concept inclusion & $C \sqsubseteq D$ &
                                                        $C^{\mathcal{I}}
                                                        \subseteq
                                                        D^{\mathcal{I}}$
    \\
    \hline
    role inclusion & $r_1 \circ ... \circ r_n \sqsubseteq r$ &
                                                               $r_1^{\mathcal{I}}
                                                               \circ
                                                               ... \circ
                                                               r_n^{\mathcal{I}}
                                                               \subseteq
                                                               r^{\mathcal{I}}$
    \\
    \hline
    
  \end{tabular}
}
\end{frame}

\begin{frame}
  \frametitle{Models}
  \begin{itemize}
  \item Interpretations and $\Sigma$-structures
  \item Model $\mathfrak{A}$ of a formula $\phi$: $\phi$ is true in
    $\mathfrak{A}$ ($\mathfrak{A} \models \phi$) 
  \item Theory $T$: set of formulas
  \item $\mathfrak{A}$ is a model of $T$ if $\mathfrak{A}$ is a model
    of all formulas in $T$
  \item Ontologies are (special kinds of) theories
  \end{itemize}
\end{frame}

\begin{frame}
  \frametitle{EL Embeddings}
  \begin{itemize}
  \item given a theory/ontology $T$ with signature $\Sigma(T)$
  \item aim: find $f_e: \Sigma(T) \mapsto \Re^n$ s.t. $f_e(\Sigma(T))$
    is a model of $T$ ($f_e(\Sigma(T)) \models T$)
    \pause
  \item more general: find an algorithm that maps symbols (signatures)
    into $\Re^n$ so that the {\em semantics} of the symbol (expressed
    through axioms and explicit in model structures) is preserved
    \begin{itemize}
    \item or: the embedding function {\em is} an interpretation function
    \end{itemize}

%     \pause
%   \item any consistent \el  theory has infinite models
%     \pause
%   \item any consistent \el  theory has models in $\mathbb{R}^n$
%     (Loewenheim-Skolem, upwards; compactness)
  \end{itemize}
\end{frame}

\begin{frame}
  \frametitle{Key idea}
  \begin{itemize}
  \item for all $r \in \Sigma(T)$ and $C \in \Sigma(T)$, define
    $f_e(r)$ and $f_e(C)$
  \item $f_e(C)$ maps to points in an open $n$-ball such that
    $f_e(C) = C^{\mathcal{I}}$:
    $C^{\mathcal{I}} = \{ x \in \mathbb{R}^n | \norm{f_e(C) - x} <
    r_e(C) \}$
    \begin{itemize}
    \item these are the {\em extension} of a class in $\Re^n$
    \end{itemize}
  \item $f_e(r)$ maps a binary relation $r$ to a vector such that
    % the set of tuples ($\mathbb{R}^n \times \mathbb{R}^n$) with
    % $f_e(r) = r^{\mathcal{I}}$:
    $r^{\mathcal{I}} = \{ (x,y) | x + f_e(r) = y \}$
    \begin{itemize}
    \item that's the TransE property for {\em individuals}
    \end{itemize}
  \item use the axioms in $T$ as constraints
  \end{itemize}
\end{frame}

\begin{frame}
  \frametitle{Algorithm}
  \begin{itemize}
  \item normalize the theory:
    \begin{itemize}
    \item every \el theory can be expressed using four normal forms
      (Baader et al., 2005)
    \end{itemize}
  \item eliminate the ABox: replace each individual symbol with a
    singleton class: $a$ becomes $\{a\}$
  \item rewrite relation assertions $r(a,b)$ and class
    assertions $C(a)$ as $\{ a \} \sqsubseteq \exists r.\{ b \}$ and
    $\{ a \} \sqsubseteq C$
    % \begin{itemize}
    % \item something to remember for the next class-vs-instance discussion?
    % \end{itemize}
  \item normalization rules to generate:
    \begin{itemize}
    \item $C \sqsubseteq D$
    \item $C \sqcap D \sqsubseteq E$
    \item $C \sqsubseteq \exists R.D$
    \item $\exists R.C \sqsubseteq D$
    \end{itemize}
  \end{itemize}
\end{frame}

\begin{frame}
  \frametitle{Algorithm: loss functions}
  \begin{equation}
    \label{eqn:nf1}
    \begin{split}
      & loss_{C \sqsubseteq D}(c,d) = \\
      & \max(0, \norm{f_\eta(c) - f_\eta(d)} + r_\eta(c) - r_\eta(d) - \gamma) \\
      & + |\norm{f_\eta(c)} - 1| + |\norm{f_\eta(d)} - 1|
    \end{split}
  \end{equation}
\end{frame}

% \begin{frame}
%   \frametitle{Algorithm: loss functions}
%   Let
%   $h=\frac{r_\eta(c)^2-r_\eta(d)^2+\norm{f_\eta(c) -
%       f_\eta(d)}^2}{2\norm{f_\eta(c) - f_\eta(d)}}$, then the center and
%   radius of the smallest $n$-ball containing the intersection of
%   $\eta(C)$ and $\eta(D)$ are
%   $f_\eta(c)+\frac{h}{\norm{f_\eta(c) -
%       f_\eta(d)}}(f_\eta(d)-f_\eta(c))$ and $\sqrt{r_\eta(c)^2-h^2}$.
% \end{frame}

\begin{frame}
  \frametitle{Algorithm: loss functions}
  \begin{equation}
    \label{eqn:nf3}
    \begin{split}
      & loss_{C \sqsubseteq \exists R.D}(c,d,r) = \\
      & \max(0, \norm{f_\eta(c) + f_\eta(r) - f_\eta(d)}  + r_\eta(c) - r_\eta(d) - \gamma) \\
      & + |\norm{f_\eta(c)} - 1| + |\norm{f_\eta(d)} - 1|
    \end{split}
  \end{equation}
\end{frame}

\begin{frame}
  \frametitle{Algorithm: loss functions}
  \begin{equation}
    \label{eqn:nf4}
    \begin{split}
      & loss_{\exists R.C \sqsubseteq D}(c,d,r) = \\
      & \max(0, \norm{f_\eta(c) - f_\eta(r) - f_\eta(d)} - r_\eta(c) - r_\eta(d) - \gamma) \\
      & + |\norm{f_\eta(c)} - 1| + |\norm{f_\eta(d)} - 1|
    \end{split}
  \end{equation}
\end{frame}

\begin{frame}
  \frametitle{Algorithm: loss functions}
  \begin{equation}
    \label{eqn:disjoint}
    \begin{split} 
      &loss_{C \sqcap D \sqsubseteq \bot}(c,d,e) = \\
      & \max(0, r_\eta(c) + r_\eta(d) - \norm{f_\eta(c) - f_\eta(d)} + \gamma) \\
      & + |\norm{f_\eta(c)} - 1| + |\norm{f_\eta(d)} - 1|
    \end{split}
  \end{equation}
\end{frame}

\begin{frame}
  \frametitle{Algorithm: loss functions}
  \centerline{\includegraphics[width=1\textwidth]{ellosses.png}}
\end{frame}

\begin{frame}
  \frametitle{EL Embeddings}
  \begin{eqnarray}
    & Male & \sqsubseteq Person \\
    & Female & \sqsubseteq Person \\
    & Father & \sqsubseteq Male \\
    & Mother & \sqsubseteq Female \\
    & Father & \sqsubseteq Parent \\
    & Mother & \sqsubseteq Parent \\
    & Female \sqcap Male & \sqsubseteq \bot \\
    & Female \sqcap Parent & \sqsubseteq Mother \\
    & Male \sqcap Parent & \sqsubseteq Father \\
    & \exists hasChild.Person & \sqsubseteq Parent \\
    & Parent & \sqsubseteq Person \\
    & Parent & \sqsubseteq \exists hasChild.\top
               \label{famlast}
               % & Person & \sqsubseteq \top
  \end{eqnarray}
\end{frame}

\begin{frame}
  \frametitle{EL Embeddings}
  \centerline{\animategraphics[loop,controls,width=.7\textwidth]{12}{embeds-frame-}{0}{99}}
  \begin{itemize}
  \item model with $\Delta = R^n$
  \item support quantifiers, negation, conjunction,...
  \end{itemize}
\end{frame}


\end{document}